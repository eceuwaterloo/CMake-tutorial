%
% (c) 2011-2016 - Eric NOULARD  <eric.noulard@gmail.com>
% This is a CMake (https://cmake.org) tutorial
% the material is open and may be found on github:
% https://github.com/TheErk/CMake-tutorial
%
% This set of slides are licensed with
% the creative commons CC-BY-SA license.
%
% For printed version uncomment trans
% If you want nice PDF presentation
% you may use Impress!ve: http://impressive.sourceforge.net/
% you use https://github.com/jberger/MakeBeamerInfo in order
% to create the impress!ve script more easily
%
\documentclass[compress,slidestop,table,usepdftitle=false
%               trans
              ]
               {beamer}

\usepackage{iwona}
\usepackage[english]{babel}
\usepackage[utf8]{inputenc}
\usepackage{xcolor}
\definecolor{cmakeblue}{RGB}{84,84,216}
\definecolor{cmakered}{RGB}{235,69,69}
\definecolor{cmakegreen}{RGB}{0,219,0}
\colorlet{cmaketimec}{green}
\colorlet{buildtimec}{red}
\colorlet{installtimec}{black}
\colorlet{cpacktimec}{blue}

\usepackage{graphicx}
\graphicspath{{figures/}{images/}}
\DeclareGraphicsExtensions{.eps,.png,.pdf,.eps,.jpg}

% Uncomment this for handout mode
\mode<handout>{
    \usepackage{pgfpages}
    \pgfpagesuselayout{2 on 1}[a4paper,border shrink=5mm]
    \setbeamercolor{background canvas}{bg=black!5}
}

\mode<presentation>{
     %\setbeamercovered{transparent}
     \setbeamercovered{invisible}
     % progressbar is a nice beamer theme by Sylvain BOUVERET
     % http://recherche.noiraudes.net/fr/LaTeX.php
     \usetheme{progressbar}
     \progressbaroptions{
               %headline=sections,
               imagename=figures/CMake-logo-triangle-small.png,
               titlepage=normal,
               frametitle=picture-section
                        }
}


\usepackage{fancyvrb}
\VerbatimFootnotes

\usepackage{ulem}
\usepackage{multirow}
\usepackage{multicol}
\usepackage{tikz}
\usetikzlibrary{arrows,shapes}
\usetikzlibrary{patterns,snakes,automata,topaths}
\usetikzlibrary{matrix,chains}
\usetikzlibrary{shadows}
\usetikzlibrary{positioning}
\usetikzlibrary{shadings}
\usetikzlibrary{calc}
\tikzstyle{na} = [baseline=-.5ex]
\tikzstyle{every picture}+=[remember picture]


%\usepackage[underline=true,rounded corners=false]{pgf-umlsd}

\usepackage[formats]{listings}
\input{lstdefine-CMake}

\lstset{escapeinside={(*@}{@*)}}
\lstset{language=CMake,
  basicstyle=\normalsize,
  keywordstyle=\bfseries\textcolor{blue},
  identifierstyle=\sffamily,
  commentstyle=\itshape\textcolor{olive},
  stringstyle=\ttfamily,
  extendedchars=false,
  showstringspaces=true,
  numbers=left,
  numberstyle=\tiny,
  stepnumber=1,
  frame=single,
  tabsize=3
}

\hypersetup{
  colorlinks=true,
  bookmarks=true,
  bookmarksopen=true,
  bookmarksopenlevel=4,
  pdfpagemode=UseOutlines,
  pdftitle={CMake [and friends] tutorial},
  pdfauthor={Eric NOULARD},
  pdfsubject={CMake},
  pdfkeywords={CMake, CPack, CTest, CDash, build systems, autotools, SCons}
}

\setcounter{tocdepth}{5}% Show up to level 4 (\paragraph) in ToC (and bookmarks)
\setcounter{secnumdepth}{5}% Show up to level 4 (\paragraph) in ToC (and bookmarks)


%
% Backslash '\'
\def\bs{\texttt{\char '134}}

\newcommand{\msgsrc}[1]{\lstinline[language=Java]!#1!}
%\newcommand{\fname}[1]{\verb+#1+}
\newcommand{\fname}[1]{\texttt{#1}}
\usepackage{xspace}
\newcommand{\cm}{CMake\xpspace}
\newcommand{\cp}{CPack\xpspace}
\newcommand{\ct}{CTest\xpspace}
\newcommand{\cd}{CDash\xpspace}

\AtBeginSubsection[]
{
   \begin{frame}
       \frametitle{Outline}
       \tableofcontents[currentsection,currentsubsection]
   \end{frame}
}

\AtBeginSection[]
{
   \begin{frame}
       \frametitle{Outline}
       \tableofcontents[currentsection]
   \end{frame}
}

\begin{document}
\pgfdeclareimage[interpolate=true,height=0.6cm]{CMakeLogo}{figures/CMake-logo-small}
\pgfdeclarelayer{background2}
\pgfdeclarelayer{background}
\pgfdeclarelayer{foreground}
\pgfsetlayers{background2,background,main,foreground}

% If you tailor the presentation to your need
% you can perfectly put your name on it.

\title{CMake tutorial}
\subtitle{and its friends CPack, CTest and CDash}
\author[Eric NOULARD - \url{eric.noulard@gmail.com}]{Eric NOULARD - \url{eric.noulard@gmail.com}\\
        \includegraphics[width=5cm]{figures/CMake-logo-small}\\
        \url{https://cmake.org}}
\date{Compiled on \today}
\subject{Slides for CMake, CPack, CTest, CDash presentation. In english.}
\keywords{CMake, CPack, CTest, CDash, build systems, autotools, SCons}

\pdfbookmark[0]{CMake tutorial}{cmaketut}
\begin{frame}
\titlepage
\begin{center}
{\tiny
This presentation is licensed

\includegraphics[width=1.2cm]{figures/by-sa}

\url{http://creativecommons.org/licenses/by-sa/3.0/us/}

\url{https://github.com/TheErk/CMake-tutorial}

Initially given by Eric Noulard for Toulibre on February, $8^{th}$ 2012.
}
\end{center}
\end{frame}

\begin{frame}
\frametitle{Thanks to\ldots}
\begin{itemize}
\item \textcolor{cmakeblue}{Kitware for making a really nice set of tools and making them open-source}
\item \textcolor{cmakered}{the CMake mailing list for its friendliness and its more than valuable source of information}
\item \textcolor{cmakegreen}{CMake developers for their tolerance when I break the dashboard or mess-up
      with the Git workflow,}
\item \textcolor{cmakeblue}{CPack users for their patience when things don't work as they \sout{should} expect}
\item \textcolor{cmakeblue}{Alan},
      \textcolor{cmakered}{Alex},
      \textcolor{cmakegreen}{Bill},
      \textcolor{cmakeblue}{Brad},
      \textcolor{cmakered}{Clint},
      \textcolor{cmakegreen}{David},
      \textcolor{cmakeblue}{Eike},
      \textcolor{cmakered}{Julien},
      \textcolor{cmakegreen}{Mathieu},
      \textcolor{cmakeblue}{Michael \& Michael},
      \textcolor{cmakered}{Stephen},
      \textcolor{cmakegreen}{Domen},
      \textcolor{cmakeblue}{and}  \textcolor{cmakered}{many} \textcolor{cmakegreen}{more}\ldots
\item \textcolor{cmakegreen}{My son Louis for the nice CPack 3D logo done with Blender.}
\item \textcolor{cmakered}{and...Toulibre for initially hosting this presention in Toulouse, France.}
\end{itemize}
\end{frame}

\part{CMake}

\section{Overview}

\begin{frame}
\frametitle{And thanks to contributors as well\ldots}

{%\setbeamerfont{block title}{size=\fontsize{6pt}{9.0}}
\begin{block}{History}
\fontsize{9pt}{11}\selectfont
This presentation  was initially made  by Eric Noulard for  a Toulibre
(\url{http://www.toulibre.fr}) given in Toulouse (France) on February,
$8^{th}$ 2012.  After that,  the source of  the presentation  has been
release                         under                        CC-BY-SA,
\url{http://creativecommons.org/licenses/by-sa/3.0/us/} and put on
\url{https://github.com/TheErk/CMake-tutorial} then contributors
stepped-in.
\end{block}
{\normalsize
Many thanks to all contributors (alphabetical order):
}
\begin{block}{Contributors}
 \fontsize{9pt}{11}\selectfont
 \textcolor{cmakeblue}{Sébastien Dinot},
 \textcolor{cmakered}{Andreas Mohr}.
\end{block}
}
\end{frame}

\section{Introduction}
\begin{frame}[fragile]
\frametitle{CMake tool sets}

\begin{block}{CMake}
CMake is a \emph{cross-platform build systems generator} which makes it easier
to build software in a unified manner on a broad set of platforms:

\includegraphics[width=0.7cm]{TuxLogo_48},
\includegraphics[width=0.7cm]{FreeBSDLogo_48},
\includegraphics[width=0.7cm]{MSWindowsLogo_48},
\includegraphics[width=0.7cm]{AppleLogo_48},
AIX, IRIX,
\includegraphics[width=0.7cm]{AndroidLogo},
\includegraphics[width=0.7cm]{AppleIOSLogo_48}
$\cdots$
\end{block}
CMake has friends softwares that may be used on their own or together:
\begin{itemize}
\item CMake: build system generator
\item CPack: package generator
\item CTest: systematic test driver
\item CDash: a dashboard collector
\end{itemize}
\end{frame}

\begin{frame}
\frametitle{Outline of Part I: CMake}
\tableofcontents[part=1]
\end{frame}

\subsection*{Introduction to build systems}
\begin{frame}
\frametitle{Build what?}
\begin{block}{Software build system}
A software build system is the usage of a [set of] tool[s] for building software applications.
\end{block}
\begin{block}{Why do we need that?}
\begin{itemize}
\pause
\item because most softwares consist of several parts that need
      some \emph{building} to put them together,
\pause
\item because softwares are written in \emph{various languages}
      that may share the same building process,
\pause
\item because we want to build \emph{the same software for various computers} {\scriptsize (PC, Macintosh, Workstation, mobile phones and other PDA, embedded computers) and systems (Windows, Linux, *BSD, other Unices (many), Android, etc\ldots)}
\end{itemize}
\end{block}
\end{frame}

\begin{frame}[fragile]
\frametitle{Programming languages}
\begin{block}{Compiled vs interpreted or what?}
Building an application requires the use
of some programming \emph{language}: Python, Java, C++, Fortran, C, Go, Tcl/Tk, Ruby, Perl, OCaml,\ldots
\end{block}

\begin{tikzpicture}[
sbase/.style={
                    % The shape:
                    rectangle,
                    % The size:
                    minimum size=2mm,
                    minimum width=1.5cm,
                    % The border:
                    thick,
                    draw=black!50!black!50,
                    % 50% red and 50% black,
                    % and that mixed with 50% white
                    % The filling:
                    top color=white,
                    % a shading that is white at the top...
                    bottom color=red!80!black!20, % and something else at the bottom
                    % Font
                                       font=\itshape\scriptsize},
scompiled/.style={sbase,
                    node distance=1cm and 1cm,
                    bottom color=green!70!black!20, % and something else at the bottom
                },
sinterpreted/.style={sbase,
                    node distance=1cm and 1cm,
                    bottom color=orange!70!black!20, % and something else at the bottom
                },
sintercomp/.style={sbase,
                   node distance=1cm and 1cm,
                   top color=orange!70!black!20, % and something else at the bottom
                   bottom color=green!70!black!20,
                },
stools/.style={sbase,
              rectangle,
              minimum width=0cm,
              bottom color=blue!80!black!20,
              font=\slshape\scriptsize
             }
                    ]
\tikzstyle{boitearrondie} = [draw,
                             dashed,
                             opacity=.5,
                             fill=blue!20,
                             rounded corners]


\onslide<1->{\node [sbase, rotate=90, minimum width=3.5cm] (language)
                                        {
                                          \begin{tabular}{c}
                                            Programming\\
                                            languages
                                          \end{tabular}
                                        };
             \node [matrix, below right=0.0 cm of language.south east] (linterp) {
               \node [sinterpreted] (python)  {Python}; \\
               \node [sinterpreted] (perl)    {Perl}; \\
               \node [sinterpreted] (matlab)  {Matlab}; \\
               \node [sintercomp] (ocaml)   {OCaml}; \\
             };
             %\node [matrix, right delimiter =\}, below=0.0cm of linterp] (lcompil) {
             \node [matrix, below=0.0cm of linterp] (lcompil) {
                \node [scompiled]    (ccxx)     {C/C++}; \\
                \node [scompiled]    (fortran) {Fortran}; \\
                \node [scompiled]    (ada) {ADA}; \\
             };
             }
\node[stools, above right=0cm and 3.5cm of linterp.west] (interpreter) {interpreter};
\node[stools, below right=0cm and 1.5cm of lcompil.south east, rotate=90] (object) {\begin{tabular}{c}
                                                                                   object\\
                                                                                   code
                                                                                  \end{tabular}
                                                                                };
\node[stools, below right=0cm and 1.5 cm of object.south west, rotate=90] (executable) {executable};
\node[stools, minimum height=2cm, right=0 cm and 2 cm of executable.south east ] (running) {\begin{tabular}{c}
                                                                                   Running\\
                                                                                   program
                                                                                  \end{tabular}};
\draw [->,thick,color=black] (linterp.east) -- (interpreter.west) node[above,midway,font=\tiny] {?byte-compile?};
\draw [->,thick,color=black] (interpreter.east) -- (running.west) node[above,midway,font=\tiny] {interprets} ;
\draw [->,thick,color=black] (lcompil.east) -- (object.north) node[above,midway,font=\tiny] {compiles};
\draw [->,thick,color=black] (object.south) -- (executable.north) node[above,midway,font=\tiny] {links};
\draw [->,thick,color=black] (executable.south) -- (running.west) node[above,midway,font=\tiny] {executes};
\begin{pgfonlayer}{background}
\onslide<2->{
\draw[boitearrondie,top color=blue,bottom color=green,middle color=red] ([xshift=0.05cm]language.south) rectangle ([xshift=0.1cm,yshift=-0.7cm]running.south east);
\pgftext[top,at={\pgfpointanchor{running}{south}}]{\pgfuseimage{CMakeLogo}};
}
\end{pgfonlayer}
\end{tikzpicture}
\end{frame}

\begin{frame}
\frametitle{Build systems: several choices}
\begin{block}{Alternatives}
CMake is not the only build system [generator]:\\
\begin{itemize}
\item (portable) hand-written Makefiles, depends on \fname{make} tool (may be \href{https://www.gnu.org/software/make/}{GNU Make}).
\item Apache \href{http://ant.apache.org/}{ant} (or \href{https://maven.apache.org/}{Maven} or \href{https://gradle.org/}{Gradle}), dedicated to Java (almost).
\item Portable IDE: Eclipse, Code::Blocks, Geany, NetBeans, \ldots
\item GNU Autotools: \href{http://www.gnu.org/software/autoconf/}{Autoconf}, Automake, Libtool.
Produce makefiles. Bourne shell needed (and M4 macro processor).
\item \href{SCons}{http://www.scons.org} only depends on Python.
\item \ldots
\end{itemize}
\end{block}
\end{frame}

\begin{frame}[fragile]
\frametitle{Build systems or build systems generator}
\begin{block}{Build systems}
A tool which builds, a.k.a. compiles, a set of source files in order to produce binary executables and libraries.
Those kind of tools usually takes as input a file (e.g. a Makefile) and while reading it issues compile commands.
The main goal of a build tool is to (re)build the minimal subset of files when something changes.
A non exhaustive list: \href{https://www.gnu.org/software/make/}{[GNU] make}, \href{https://ninja-build.org/}{ninja}, \href{https://github.com/Microsoft/msbuild}{MSBuild}, \href{http://www.scons.org}{SCons}, \href{http://ant.apache.org/}{ant}, \ldots
\end{block}
A \textbf{Build systems generator} is a tool which generates files for a particular build system. e.g. \href{http://cmake.org}{CMake} or \href{http://www.gnu.org/software/autoconf/}{Autotools}.
\end{frame}

\begin{frame}
  \frametitle{What build systems do?}
  \begin{block}{Targets and sources}
    The main feature of a build system is to offer a way to describe how a target (executable, PDF, shared library\ldots) is built from its sources (set of object files and/or libraries, a latex or rst file, set of C/C++/Fortran files\ldots). Basically a \emph{target} \textbf{depends} on one or several {sources} and one can run a set of \textbf{commands} in order to built the concerned \emph{target} from its \emph{sources}.
  \end{block}
  The main goals/features may be summarized as:
  \begin{itemize}
  \item describe dependency graph between sources and targets
  \item associate one or several commands to \emph{rebuilt} target from source(s)
  \item issue the \emph{minimal} set of commands in order to rebuild a target
  \end{itemize}
\end{frame}

\begin{frame}[label=samplemakefile,fragile]
  \frametitle{A sample \texttt{Makefile} for \texttt{make}}
\lstinputlisting[basicstyle=\tiny,numbers=left,firstline=2,language=make]{examples/totally-free/Makefile.manual}
\end{frame}

\begin{frame}
\frametitle{Comparisons and [success] stories}

\begin{alertblock}{Disclaimer}
This presentation is biased. \emph{I mean totally}.

I am a big CMake fan, I did contribute to CMake, thus I'm not
impartial \emph{at all}. But I will be ready to discuss why CMake
is the greatest build system out there :-)
\end{alertblock}

Go and forge your own opinion:
\begin{itemize}
\item Bare list: \url{http://en.wikipedia.org/wiki/List_of_build_automation_software}
\item A comparison: \url{http://www.scons.org/wiki/SconsVsOtherBuildTools}
\item KDE success story (2006): ``\textsl{Why the KDE project switched to CMake -- and how}''
     \url{http://lwn.net/Articles/188693/}
\end{itemize}
\end{frame}

\begin{frame}
\frametitle{CMake/Auto[conf\textbar make] on OpenHub}
\begin{center}
\includegraphics[width=0.9\textwidth,height=0.50\textheight]{compare_cmake_autotools_ohlo_color_transparent}
\end{center}
\begin{block}{\url{https://www.openhub.net/languages/compare}}
Language comparison of CMake to automake and
autoconf showing the percentage of developers commits that modify a
source file of the respective language (data from 2012).
\end{block}
\end{frame}

\begin{frame}
\frametitle{CMake/Autoconf/Gradle on Google Trend}
\begin{center}
\includegraphics[width=1.0\textwidth,height=0.50\textheight]{CMakeAutoconfGradle-trend-2016}
\end{center}
\begin{block}{\url{https://www.google.com/trends}}
Scale is based on the average worldwide request traffic searching for CMake, Autoconf and Gradle in all years (2004--now).
\end{block}
\end{frame}

\section{Basic CMake usage}

\begin{frame}
\frametitle{A build system generator}
\begin{itemize}
\item CMake is a \emph{generator}: it generates \emph{native} build systems files (Makefile, Ninja, IDE project files [XCode, CodeBlocks, Eclipse CDT, Codelite, Visual Studio, Sublime Text\ldots], \ldots),
\item CMake scripting language (declarative) is used to describe the build,
\item The developer edits \fname{CMakeLists.txt}, invokes CMake but
      should \emph{never} edit the generated files,
\item CMake may be (automatically) re-invoked by the build system,
\item CMake has friends who may be very handy (CPack, CTest, CDash)
\end{itemize}
\end{frame}

\begin{frame}
\frametitle{The CMake workflow}
\begin{enumerate}
\onslide<2->{\item \emph{CMake time}: CMake is running \& processing \fname{CMakeLists.txt}}
\onslide<3->{\item \emph{Build time}: the build tool runs and invokes (at least) the compiler}
\onslide<4->{\item \emph{Install time}: the compiled binaries are installed

             i.e. from build area to an install location.
            }
\onslide<5->{\item \emph{CPack time}: CPack is running for building package}
\onslide<6->{\item \emph{Package Install time}: the package (from previous step) is installed}
\end{enumerate}
\onslide<1->{
\begin{alertblock}{When do things take place?}
CMake is a \emph{generator} which means it does not compile (i.e. build) the sources,
the underlying build tool (make, Ninja, XCode, Visual Studio\ldots) does.
\end{alertblock}
}
\end{frame}

\begin{frame}[label=cmakeworkflow]
\frametitle{The CMake workflow (pictured)}
\begin{tikzpicture}[
sbase/.style={      % The shape:
                    rectangle,
                    % The size:
                    minimum size=2mm,
                    minimum width=2.0cm,
                    minimum width=0.5cm,
                    % The border:
                    thick,
                    draw=black!50!black!50,
                    % 50% red and 50% black,
                    % and that mixed with 50% white
                    % The filling:
                    top color=white,
                    % a shading that is white at the top...
                    bottom color=red!80!black!20, % and something else at the bottom
                    % Font
                    font=\itshape\scriptsize}
                    ]
\tikzstyle{edited} = [sbase,
                      draw,
                      bottom color=green!80!black!20,
                      %opacity=.5,
                      %fill=green!20,
                      rounded corners]
\tikzstyle{generated} = [sbase,
                      draw,
                      %dashed,
                      bottom color=red!80!black!20,
                      %opacity=.5,
                      %fill=green!20,
                      rounded corners]
\tikzstyle{installed} = [sbase,
                      draw,
                      bottom color=blue!80!black!20,
                      %opacity=.5,
                      %fill=green!20,
                      rounded corners]
\tikzstyle{pkg} = [installed,
                  dashed,
                  general shadow={fill=blue!60!black!40,shadow scale=1.05,shadow xshift=+2pt,shadow yshift=-2pt}
                  ]
\onslide<1->{
\node[edited] (cmakelists) {\begin{tabular}{c}
                            CMakeLists.txt
                           \end{tabular}
                            };
\node[edited,below=1cm and 0cm of cmakelists.south] (sourcefiles)
                           {\begin{tabular}{c}
                            Source files
                           \end{tabular}
                           };
            }
\onslide<2->{
\node[generated,right=0cm and 1.5cm of cmakelists.east] (projectfiles)
                           {\begin{tabular}{c}
                            Project file(s),\\
                            Makefiles, \ldots
                           \end{tabular}
                           };
\node[generated,below right=2cm and 1.5cm of cmakelists.east] (gensourcefiles) {\begin{tabular}{c}
                            Generated \\
                            Sources files
                           \end{tabular}
                           };
}
\onslide<3->{
\node[generated,right=0cm and 1.5cm of projectfiles.east] (objectfiles)  {\begin{tabular}{c}
                             Object files
                             \end{tabular}
                            };
}
\onslide<5->{
\node[pkg, below left=1.7cm and 0cm of sourcefiles.south] (spackage)
                            {\begin{tabular}{c}
                              Source \\
                              package
                            \end{tabular}
                            };
}
\node[above right=0.5cm and 2.0cm of spackage.east] (legendL){};
\node[right=0cm and 2.3cm of legendL.east] (legendR){};
\onslide<5->{
\node[pkg, below right=-1.5cm and 7cm of spackage.east] (bpackage)
                            {\begin{tabular}{c}
                              Binary \\
                              package
                            \end{tabular}
                            };
}
\onslide<6->{
\node[installed, below=0.7cm and 0cm of bpackage.south] (ipackage)
                            {\begin{tabular}{c}
                              Installed \\
                              package
                            \end{tabular}
                            };

}
\onslide<4->{
\node[installed, above=0.7cm and 0cm of bpackage.north] (binstalled)
                            {\begin{tabular}{c}
                              Installed \\
                              files
                            \end{tabular}
                            };
}
\tikzstyle{cmaketime} = [-latex,thick,color=green]
\tikzstyle{buildtime} = [-latex,thick,color=red]
\tikzstyle{installtime} = [-latex,thick,color=black]
\tikzstyle{cpacktime} = [-latex,thick,color=blue]
\tikzstyle{packageinstalltime} = [-latex,thick,color=black,dashed]
\onslide<2->{
\draw [cmaketime]           ([yshift=0.0cm]legendL.south) -- ([yshift=0.0cm]legendR.south)  node[above=-2pt,midway,font=\tiny] {CMake time};
}
\onslide<3->{
\draw [buildtime]          ([yshift=-0.5cm]legendL.south) -- ([yshift=-0.5cm]legendR.south) node[above=-2pt,midway,font=\tiny] {Build time};
}
\onslide<4->{
\draw [installtime]        ([yshift=-1.0cm]legendL.south) -- ([yshift=-1.0cm]legendR.south) node[above=-2pt,midway,font=\tiny] {Install time};
}
\onslide<5->{
\draw [cpacktime]          ([yshift=-1.5cm]legendL.south) -- ([yshift=-1.5cm]legendR.south) node[above=-2pt,midway,font=\tiny] {CPack time};
}
\onslide<6->{
\draw [packageinstalltime] ([yshift=-2.0cm]legendL.south) -- ([yshift=-2.0cm]legendR.south) node[above=-2pt,midway,font=\tiny] {Package Install time};
}
\begin{pgfonlayer}{background}
\onslide<2->{
\draw [cmaketime] (cmakelists) -- (projectfiles);
\draw [cmaketime] (cmakelists) -- (gensourcefiles);
\draw [cmaketime] (sourcefiles) -- (gensourcefiles);
}
\onslide<5->{
\draw [cpacktime] (cmakelists) -- (spackage);
\draw [cpacktime] (sourcefiles) -- (spackage);
\draw [cpacktime] ([xshift=-0.2cm]binstalled.south) -- ([xshift=-0.2cm]bpackage.north);
}
\onslide<3->{
\draw [buildtime] (sourcefiles) -- (objectfiles);
\draw [buildtime] (gensourcefiles) -- (objectfiles);
\draw [buildtime] (projectfiles) -- (gensourcefiles);
}
\onslide<4->{
\draw [installtime] (objectfiles) -- (binstalled);
\draw [installtime] (gensourcefiles) -- (binstalled);
\draw [installtime] (sourcefiles) -- (binstalled);
}
\onslide<6->{
\draw [packageinstalltime] (bpackage.south) -- (ipackage.north);
}
\end{pgfonlayer}
\end{tikzpicture}
\end{frame}

\begin{frame}[fragile]
\frametitle{Building an executable}
\begin{lstlisting}[basicstyle=\scriptsize,caption=Building a simple program]
cmake_minimum_required (VERSION 3.0)
# This project use C source code
project (TotallyFree C)
set(CMAKE_C_STANDARD 99)
set(CMAKE_C_EXTENSIONS False)
# build executable using specified list of source files
add_executable(Acrolibre acrolibre.c)
\end{lstlisting}
CMake scripting language is [mostly] declarative.
It has \emph{commands} which are documented from within CMake:
{\tiny
\begin{verbatim}
 $ cmake --help-command-list | wc -l
 117
 $ cmake --help-command add_executable
...
  add_executable
       Add an executable to the project using the specified source files.
\end{verbatim}
}
\end{frame}

\begin{frame}[fragile,allowframebreaks]
\frametitle{Builtin documentation}
\begin{Verbatim}[commandchars=\\\{\},fontsize=\tiny,numbers=left,frame=topline,label=CMake builtin doc for 'project' command]
  $ cmake --help-command project
project
-------

Set a name, version, and enable languages for the entire project.

 project(<PROJECT-NAME> [LANGUAGES] [<language-name>...])
 project(<PROJECT-NAME>
         [VERSION <major>[.<minor>[.<patch>[.<tweak>]]]]
         [LANGUAGES <language-name>...])

Sets the name of the project and stores the name in the
``PROJECT_NAME`` variable.
[...]
       Optionally you can specify which languages your project supports.
       Example languages are \textcolor{cmakeblue}{CXX} (i.e.  C++), \textcolor{cmakeblue}{C}, \textcolor{cmakeblue}{Fortran}, etc.  By \textcolor{cmakered}{default C} \textcolor{cmakered}{and CXX are enabled}.
       E.g.  if you do  not have a C++ compiler, you can disable the check  for it by  explicitly
       listing the languages  you want  to support, e.g.  C. By using the special language "\textcolor{cmakeblue}{NONE}"
       all checks for any language can be disabled.
\end{Verbatim}
Online doc : \url{https://cmake.org/documentation/}\\
Unix Manual: {\scriptsize \texttt{cmake-variables(7)}, \texttt{cmake-commands(7)}, \texttt{cmake-xxx(7)}, \ldots}\\
All doc generated using \href{http://www.sphinx-doc.org/en/stable/builders.html}{Sphinx},
QtHelp file as well:\\
\begin{small}
\begin{enumerate}
\item get QtHelp file from CMake: \url{https://cmake.org/cmake/help/v3.6/CMake.qch}

      and copy it to \texttt{CMake-tutorial/examples/}
\item use CMake.qhcp you may find in the source of this tutorial:

      \texttt{CMake-tutorial/examples/CMake.qhcp}
\item compile QtHelp collection file:

      \texttt{qcollectiongenerator CMake.qhcp -o CMake.qhc}
\item display it using Qt Assistant:

  \texttt{assistant -collectionFile CMake.qhc}
\end{enumerate}
\end{small}
\end{frame}

\defverbatim[colored]{\genmake}{
\begin{Verbatim}[commandchars=\\\{\},fontsize=\tiny,numbers=left,frame=topline,label=Building with make]
 $ ls totally-free
 acrolibre.c  CMakeLists.txt
 $ mkdir build
 $ cd build
 $ cmake ../totally-free
 -- The C compiler identification is GNU 4.6.2
 -- Check for working C compiler: /usr/bin/gcc
 -- Check for working C compiler: /usr/bin/gcc -- works
 ...
 $ \textcolor{cmakeblue}{make}
 ...
  [100%] Built target Acrolibre
 $ ./Acrolibre toulibre
\end{Verbatim}
 }

\defverbatim[colored]{\genninja}{
\begin{Verbatim}[commandchars=\\\{\},fontsize=\tiny,numbers=left,frame=topline,label=Building with ninja]
 $ ls totally-free
 acrolibre.c  CMakeLists.txt
 $ \textcolor{cmakered}{mkdir build-ninja}
 $ \textcolor{cmakered}{cd build-ninja}
 $ cmake \textcolor{cmakered}{-GNinja} ../totally-free
 -- The C compiler identification is GNU 4.6.2
 -- Check for working C compiler: /usr/bin/gcc
 -- Check for working C compiler: /usr/bin/gcc -- works
 ...
 $ \textcolor{cmakered}{ninja}
 ...
 [6/6] Linking C executable Acrodictlibre
 $ ./Acrolibre toulibre
\end{Verbatim}
}

\defverbatim[colored]{\gencrosswin}{
\begin{Verbatim}[commandchars=\\\{\},fontsize=\tiny,numbers=left,frame=topline,label=Building with cross-compiler]
 $ ls totally-free
 acrolibre.c  CMakeLists.txt
 $ \textcolor{cmakegreen}{mkdir build-win32}
 $ \textcolor{cmakegreen}{cd build-win32}
 $ cmake \textcolor{cmakegreen}{-DCMAKE_TOOLCHAIN_FILE=../totally-free/Toolchain-cross-linux.cmake} ../totally-free
 -- The C compiler identification is GNU 6.1.1
 -- Check for working C compiler: /usr/bin/i686-w64-mingw32-gcc
 ...
 $ \textcolor{cmakegreen}{make}
 ...
 [100%] Linking C executable Acrolibre\textcolor{cmakegreen}{.exe}
 [100%] Built target Acrolibre
 $ ./Acrolibre toulibre
\end{Verbatim}
}

\begin{frame}[fragile]
\frametitle{Generating \& building}
\only<3>{\textcolor{cmakegreen}{Cross-}}Building with CMake and \only<1,3>{\textcolor{cmakeblue}{make}}\only<2>{\textcolor{cmakered}{ninja}} is easy:
\begin{center}
\only<1>{\genmake}\only<2>{\genninja}\only<3>{\gencrosswin}
\end{center}
\onslide<1->{
\begin{alertblock}{Source tree vs Build tree}
Even the most simple project should never mix-up sources
with generated files. CMake supports \emph{out-of-source} build.
\end{alertblock}}
\end{frame}

\begin{frame}[fragile]
\frametitle{Always build out-of-source}
\begin{alertblock}{Out-of-source is better}
People are lazy (me too) and they think that because
building in source is possible and authorizes less typing
they can get away with it.
In-source build is a \emph{BAD} choice.
\end{alertblock}
Out-of-source build is \emph{always} better because:
\begin{enumerate}
\pause
\item Generated files are separated from manually edited ones

      (thus you don't have to clutter your favorite VCS ignore files).
\pause
\item You can have several build trees for the same source tree
\pause
\item This way it's always safe to completely delete the build tree
      in order to do a clean build
\end{enumerate}
\end{frame}

\begin{frame}[fragile]
\frametitle{Building program + autonomous library}
\onslide<2->{
\begin{block}{Conditional build}
We want to keep a version of our program that can be compiled
and run without the new Acrodict library \emph{and} the new version which
uses the library.
\end{block}}
We now have the following set of files in our source tree:
\begin{itemize}
\item \fname{acrolibre.c}, the main C program
\item \fname{acrodict.h}, the Acrodict library header
\item \fname{acrodict.c}, the Acrodict library source
\item \fname{CMakeLists.txt}, the soon to be updated CMake input file
\end{itemize}
\end{frame}

\begin{frame}[fragile]
\setlength{\columnsep}{0.8cm}
\vspace*{-0.5cm}
\begin{center}
The main program source
\end{center}
\begin{multicols}{2}
\begin{lstlisting}[basicstyle=\tiny,language=C,breaklines=true]
#include <stdlib.h>
#include <stdio.h>
#include <strings.h>
#ifdef USE_ACRODICT
#include "acrodict.h"
#endif
int main(int argc, char* argv[]) {

  const char * name;
#ifdef USE_ACRODICT
  const acroItem_t*  item;
#endif

  if (argc < 2) {
    fprintf(stderr,"%s: you need one argument\n",argv[0]);
    fprintf(stderr,"%s <name>\n",argv[0]);
    exit(EXIT_FAILURE);
  }
  name = argv[1];

#ifndef USE_ACRODICT
  if (strcasecmp(name,"toulibre")==0) {
    printf("Toulibre is a french organization promoting FLOSS.\n");
  }
#else
  item = acrodict_get(name);
  if (NULL!=item) {
    printf("%s: %s\n",item->name,item->description);
  } else if (item=acrodict_get_approx(name)) {
    printf("<%s> is unknown may be you mean:\n",name);
    printf("%s: %s\n",item->name,item->description);
  }
#endif
  else {
    printf("Sorry, I don't know: <%s>\n",name);
    return EXIT_FAILURE;
  }
  return EXIT_SUCCESS;
}
\end{lstlisting}
\end{multicols}
\end{frame}

\begin{frame}[fragile]
\setlength{\columnsep}{0.8cm}
\vspace*{-0.5cm}
\begin{center}
The library source
\end{center}
\begin{multicols}{2}
\begin{lstlisting}[basicstyle=\tiny,language=C,breaklines=true]
#ifndef ACRODICT_H
#define ACRODICT_H
typedef struct acroItem {
  char* name;
  char* description;
} acroItem_t;

const acroItem_t*
acrodict_get(const char* name);
#endif
\end{lstlisting}
\begin{lstlisting}[basicstyle=\tiny,language=C,breaklines=true]
#include <stdlib.h>
#include <string.h>
#include "acrodict.h"
static const acroItem_t acrodict[] = {
  {"Toulibre", "Toulibre is a french organization promoting FLOSS"},
  {"GNU", "GNU is Not Unix"},
  {"GPL", "GNU general Public License"},
  {"BSD", "Berkeley Software Distribution"},
  {"CULTe","Club des Utilisateurs de Logiciels libres et de gnu/linux de Toulouse et des environs"},
  {"Lea", "Lea-Linux: Linux entre ami(e)s"},
  {"RMLL","Rencontres Mondiales du Logiciel Libre"},
  {"FLOSS","Free Libre Open Source Software"},
  {"",""}};
const acroItem_t*
acrodict_get(const char* name) {
  int current =0;
  int found   =0;
  while ((strlen(acrodict[current].name)>0) && !found) {
    if (strcasecmp(name,acrodict[current].name)==0) {
      found=1;
    } else {
      current++;
    }
  }
  if (found) {
    return &(acrodict[current]);
  } else {
    return NULL;
  }
}
\end{lstlisting}
\end{multicols}
\end{frame}

\againframe{samplemakefile}

\begin{frame}[fragile]
\frametitle{Building a library}
\vspace*{-0.5cm}
\begin{lstlisting}[basicstyle=\tiny,caption=Building a simple program + shared library]
cmake_minimum_required (VERSION 3.0)
project (TotallyFree C)
(*@\tikz[na] \coordinate(PdstC99);@*)set(CMAKE_C_STANDARD 99)
set(CMAKE_C_EXTENSIONS False)
add_executable(Acrolibre acrolibre.c)
(*@\tikz[na] \coordinate(PdstVarDef);@*)set(LIBSRC acrodict.c acrodict.h)
add_library(acrodict ${LIBSRC})(*@\tikz[na] \coordinate(PdstLibDef);@*)
add_executable(Acrodictlibre acrolibre.c)
(*@\tikz[na] \coordinate(PdstLinkOpt);@*)target_link_libraries(Acrodictlibre acrodict) (*@\label{linkline}@*)
(*@\tikz[na] \coordinate(PdstCompilOpt);@*)set_target_properties(Acrodictlibre PROPERTIES COMPILE_FLAGS "-DUSE_ACRODICT") (*@\label{propdef}@*)
\end{lstlisting}
\begin{itemize}
\item<1-> \tikz[na] \coordinate(PsrcC99);we precise that we want to compile with C99 flags
\item<2-> \tikz[na] \coordinate(PsrcVarDef);we define a variable and ask to build a library\tikz[na] \coordinate(PsrcLibDef);
\item<3-> \tikz[na] \coordinate(PsrcLinkOpt);we link an executable to our library %(line \ref{linkline})
\item<4-> \tikz[na] \coordinate(PsrcCompilOpt);we compile the source files of a particular target with specific compiler options
\end{itemize}

\begin{tikzpicture}[overlay]
  \only<1>{
    \path[->, red, thick] ($(PsrcC99)+(-0.5,0)$) edge[bend left] (PdstC99);
  }
  \only<2>{
    \path[->, red, thick] ($(PsrcVarDef)+(-0.5,0)$) edge[bend left] ($(PdstVarDef)+(0,-0.1)$);
  }
  \only<3>{
    \path[->, red, thick] ($(PsrcLinkOpt)+(-0.5,0)$) edge[bend left] ($(PdstLinkOpt)+(0,0)$);
  }
  \only<4>{
    \path[->, red, thick] ($(PsrcCompilOpt)+(-0.5,0)$) edge[bend left] ($(PdstCompilOpt)+(0,0)$);
  }
\end{tikzpicture}

\end{frame}

\begin{frame}[fragile,allowframebreaks]
\frametitle{Building a library - continued}

\begin{block}{And it builds...}
All in all CMake generates appropriate Unix makefiles which build
all this smoothly.
\end{block}
\begin{Verbatim}[fontsize=\tiny,numbers=left,frame=topline,label=CMake + Unix Makefile]
$ make
[ 33%] Building C object CMakeFiles/acrodict.dir/acrodict.c.o
Linking C shared library libacrodict.so
[ 33%] Built target acrodict
[ 66%] Building C object CMakeFiles/Acrodictlibre.dir/acrolibre.c.o
Linking C executable Acrodictlibre
[ 66%] Built target Acrodictlibre
[100%] Building C object CMakeFiles/Acrolibre.dir/acrolibre.c.o
Linking C executable Acrolibre
[100%] Built target Acrolibre
$ ls -F
Acrodictlibre*  CMakeCache.txt  cmake_install.cmake  Makefile
Acrolibre*      CMakeFiles/     libacrodict.so*
\end{Verbatim}

\begin{block}{And it works...}
We get the two different variants of our program, with varying capabilities.
\end{block}
\begin{multicols}{2}
\begin{Verbatim}[fontsize=\tiny,numbers=left]
$ ./Acrolibre toulibre
Toulibre is a french organization promoting FLOSS.
$ ./Acrolibre FLOSS
Sorry, I don't know: <FLOSS>
$ ./Acrodictlibre FLOSS
FLOSS: Free Libre Open Source Software
\end{Verbatim}
%$
\begin{Verbatim}[fontsize=\tiny,]
$ make help
The following are some of the valid targets
    for this Makefile:
... all (the default if no target is provided)
... clean
... depend
... Acrodictlibre
... Acrolibre
... acrodict
...
\end{Verbatim}
%$
Generated \fname{Makefile}s has several builtin targets besides the
expected ones:
\begin{itemize}
\item one per target (library or executable)
\item clean, all
\item more to come \ldots
\end{itemize}
\end{multicols}

\begin{alertblock}{And it is homogeneously done whatever the generator...}
  The obtained build system contains the same set of targets whatever the combination
  of generator and [cross-]compiler used: Makefile+gcc, Ninja+clang, XCode, Visual Studio, etc\ldots
\end{alertblock}
\end{frame}

\begin{frame}[fragile]
\frametitle{User controlled build option}
\begin{alertblock}{User controlled option}
Maybe our users don't want the acronym dictionary support.
We can use CMake \lstinline!OPTION! command.
\end{alertblock}
\begin{lstlisting}[basicstyle=\tiny,caption=User controlled build option]
cmake_minimum_required (VERSION 3.0)
# This project use C source code
project (TotallyFree C)
# Build option with default value to ON
option(WITH_ACRODICT "Include acronym dictionary support" ON)
set(BUILD_SHARED_LIBS true)
# build executable using specified list of source files
add_executable(Acrolibre acrolibre.c)
if (WITH_ACRODICT)
   set(LIBSRC acrodict.h acrodict.c)
   add_library(acrodict ${LIBSRC})
   add_executable(Acrodictlibre acrolibre.c)
   target_link_libraries(Acrodictlibre acrodict)
   set_target_properties(Acrodictlibre PROPERTIES COMPILE_FLAGS "-DUSE_ACRODICT")
endif(WITH_ACRODICT)
\end{lstlisting}
%$
\end{frame}


\begin{frame}[fragile,allowframebreaks]
  \frametitle{Too much keyboard, time to click?}

\begin{block}{CMake comes with severals tools}
  A matter of choice / taste:
  \vspace*{0.4cm}
\begin{itemize}
\item a command line: \fname{cmake}
\item a curses-based TUI: \fname{ccmake}
\item a Qt-based GUI: \fname{cmake-gui}
\end{itemize}
\end{block}

\begin{alertblock}{Calling convention}
All tools expect to be called with a single argument
which may be interpreted in 2 different ways.
\vspace*{0.4cm}
\begin{itemize}
\item path to the source tree, e.g.: \fname{cmake /path/to/source}
\item path to an \alert{existing} build tree, e.g.: \fname{cmake-gui .}
\end{itemize}
\end{alertblock}

\begin{center}
\fname{ccmake} : the curses-based TUI (demo)

\includegraphics[width=0.8\textwidth]{ccmake-1}

Here we can choose to toggle the \lstinline!WITH_ACRODICT OPTION!.
\end{center}

\begin{center}
\fname{cmake-gui} : the Qt-based GUI (demo)

\includegraphics[width=0.7\textwidth]{cmake-gui-1}

Again, we can choose to toggle the \lstinline!WITH_ACRODICT OPTION!.
\end{center}
\end{frame}

\begin{frame}[fragile]
\frametitle{Remember CMake is a build \alert{generator}?}
The number of active generators depends on the platform
we are running on {Unix}, \textcolor{cmakered}{Apple},
\textcolor{cmakeblue}{Windows}:
\begin{multicols}{2}
\begin{Verbatim}[commandchars=\\\{\},fontsize=\scriptsize,numbers=left]
 \textcolor{cmakeblue}{Borland Makefiles}
 \textcolor{cmakeblue}{MSYS Makefiles}
 \textcolor{cmakeblue}{MinGW Makefiles}
 \textcolor{cmakeblue}{NMake Makefiles}
 \textcolor{cmakeblue}{NMake Makefiles JOM}
 Unix Makefiles
 \textcolor{cmakeblue}{Visual Studio 10}
 \textcolor{cmakeblue}{Visual Studio 10 IA64}
 \textcolor{cmakeblue}{Visual Studio 10 Win64}
 \textcolor{cmakeblue}{Visual Studio 11}
 \textcolor{cmakeblue}{Visual Studio 11 Win64}
 \textcolor{cmakeblue}{Visual Studio 6}
 \textcolor{cmakeblue}{Visual Studio 7}
 \textcolor{cmakeblue}{Visual Studio 7 .NET 2003}
 \textcolor{cmakeblue}{Visual Studio 8 2005}
 \textcolor{cmakeblue}{Visual Studio 8 2005 Win64}
 \textcolor{cmakeblue}{Visual Studio 9 2008}
 \textcolor{cmakeblue}{Visual Studio 9 2008 IA64}
 \textcolor{cmakeblue}{Visual Studio 9 2008 Win64}
 \textcolor{cmakeblue}{Watcom WMake}
 \textcolor{cmakeblue}{CodeBlocks - MinGW Makefiles}
 \textcolor{cmakeblue}{CodeBlocks - NMake Makefiles}
 CodeBlocks - Unix Makefiles
 \textcolor{cmakeblue}{Eclipse CDT4 - MinGW Makefiles}
 \textcolor{cmakeblue}{Eclipse CDT4 - NMake Makefiles}
 Eclipse CDT4 - Unix Makefiles
 KDevelop3
 KDevelop3 - Unix Makefiles
 \textcolor{cmakered}{XCode}
 Ni\textcolor{cmakeblue}{nj}\textcolor{cmakered}{a}
\end{Verbatim}

\end{multicols}
\end{frame}

\begin{frame}[fragile]
\frametitle{Equally simple on other platforms}
It is as easy for a Windows build, however
names for executables and libraries are computed
in a
\textcolor{cmakeblue}{platform}
\textcolor{cmakered}{specific}
\textcolor{cmakegreen}{way}.
\begin{center}
\begin{Verbatim}[commandchars=\\\{\},fontsize=\tiny,numbers=left,frame=topline,label=CMake + MinGW Makefile]
 $ ls totally-free
 acrodict.h acrodict.c acrolibre.c  CMakeLists.txt
 $ mkdir build-win32
 $ cd build-win32
 $ cmake -DCMAKE_TOOLCHAIN_FILE=../totally-free/Toolchain-cross-linux.cmake ../totally-free
 ...
 $ make
 Scanning dependencies of target acrodict
 \textcolor{cmakeblue}{[ 33%] Building C object CMakeFiles/acrodict.dir/acrodict.c.obj}
 \textcolor{cmakegreen}{Linking C shared library libacrodict.dll}
 \textcolor{cmakegreen}{Creating library file: libacrodict.dll.a}
 [ 33%] Built target acrodict
 Scanning dependencies of target Acrodictlibre
 \textcolor{cmakeblue}{[ 66%] Building C object CMakeFiles/Acrodictlibre.dir/acrolibre.c.obj}
 \textcolor{cmakered}{Linking C executable Acrodictlibre.exe}
 [ 66%] Built target Acrodictlibre
 Scanning dependencies of target Acrolibre
 \textcolor{cmakeblue}{[100%] Building C object CMakeFiles/Acrolibre.dir/acrolibre.c.obj}
 \textcolor{cmakered}{Linking C executable Acrolibre.exe}
 [100%] Built target Acrolibre
\end{Verbatim}
\end{center}
\end{frame}

\begin{frame}[fragile]
\frametitle{Installing things}
\begin{block}{Install}
Several parts or the software may need to be installed:
this is controlled by the CMake \lstinline!install! command.

\alert{Remember \fname{cmake --help-command install}!!}
\end{block}
\begin{lstlisting}[basicstyle=\tiny,caption=install command examples]
...
add_executable(Acrolibre acrolibre.c)
install(TARGETS Acrolibre DESTINATION bin)
if (WITH_ACRODICT)
  ...
  install(TARGETS Acrodictlibre acrodict
           RUNTIME DESTINATION bin
           LIBRARY DESTINATION lib
           ARCHIVE DESTINATION lib/static)
  install(FILES acrodict.h DESTINATION include)
endif(WITH_ACRODICT)
\end{lstlisting}
\end{frame}


\againframe<6>{cmakeworkflow}

\begin{frame}[fragile]
\frametitle{The install target}
\begin{block}{Install target}
The \fname{install} target of the underlying build tool (in our case \fname{make})
appears in the generated build system as soon as some \lstinline!install!
commands are used in the \fname{CMakeLists.txt}.
\end{block}
\begin{Verbatim}[commandchars=\\\{\},fontsize=\scriptsize,numbers=left]
$ make DESTDIR=/tmp/testinstall install
[ 33%] Built target acrodict
[ 66%] Built target Acrodictlibre
[100%] Built target Acrolibre
Install the project...
-- Install configuration: ""
-- Installing: /tmp/testinstall/bin/Acrolibre
-- Installing: /tmp/testinstall/bin/Acrodictlibre
-- Removed runtime path from "/tmp/testinstall/bin/Acrodictlibre"
-- Installing: /tmp/testinstall/lib/libacrodict.so
-- Installing: /tmp/testinstall/include/acrodict.h
$
\end{Verbatim}
\end{frame}


\begin{frame}[fragile]
\frametitle{Summary}
\begin{block}{CMake basics}
Using CMake basics we can already do a lot of things with minimal writing.
\end{block}
\begin{itemize}
\item Write simple build specification file: \fname{CMakeLists.txt}
\item Discover compilers (C, C++, Fortran)
\item Build executable and library (shared or static) in a cross-platform manner
\item Package the resulting binaries with CPack
\item Run systematic tests with CTest and publish them with CDash
\end{itemize}
\end{frame}

\begin{frame}[fragile]
\frametitle{Seeking more information or help}
There are several places you can go by yourself:
\begin{enumerate}
\item {\tiny(re-)}Read the documentation: \url{https://cmake.org/documentation}
\item Read the FAQ: \url{https://cmake.org/Wiki/CMake_FAQ}
\item Read the Wiki: \url{https://cmake.org/Wiki/CMake}
\item Ask on the Mailing List: \url{https://cmake.org/mailing-lists}
\item Browse the built-in help:

     \fname{man cmake-xxxx}\\
     \fname{cmake --help-xxxxx}\\
     \fname{assistant -collectionFile examples/CMake.qhc}
\end{enumerate}
\end{frame}
\end{document}
